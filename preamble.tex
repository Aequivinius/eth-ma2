\usepackage{wasysym} % for male and female symbols

\usepackage[utf8]{inputenc} % Ensures UTF-8 encoding for special characters
\usepackage[ngerman]{babel} % Adapts language elements (e.g., "Abbildung" instead of "Figure") to German

% Formatting and Layout
\usepackage{geometry} % Adjusts page margins and layout
\usepackage{tcolorbox} % Creates colored boxes for emphasis or notes
\usepackage{mdframed} % Allows framed environments with background colors
\usepackage{multicol} % switch between single and two columns

% Math and Symbols
\usepackage{amsmath} % Provides advanced mathematical typesetting

% Tables and Colors
\usepackage{booktabs} % Enhances table formatting for professional appearance
\usepackage{graphicx} % Enables the inclusion of images
\usepackage[table,xcdraw]{xcolor} % Adds colors to tables and drawings

% Listings and Code
\usepackage{listings} % Formats code listings in a basic way
\usepackage{minted} % Formats code listings with syntax highlighting (requires Pygments)

% Text Manipulation
\usepackage{soul} % Allows text highlighting, strikethrough, etc.
\usepackage{emoji} % Enables inserting emojis into the document

% Lists and Enumerations
\usepackage{enumitem} % Allows me to make smore compact lists

% Figures and Captions
\usepackage{subcaption} % Supports subfigures with individual captions

% External Content and Hyperlinks
\usepackage{pdfpages} % Includes external PDFs in the document
\usepackage{hyperref} % Enables clickable links and \url command

% Miscellaneous
\usepackage{jupynotex} % For integrating Jupyter Notebook outputs into LaTeX

\geometry{
 a4paper,
 total={170mm,257mm},
 left=25mm,
 top=20mm,
 }

\newcommand\todo[1]{\noindent\textcolor{orange}{\emoji{warning} #1}}
\newcommand{\singleslide}[1]{
    \begin{tcolorbox}[colframe=black!75, width=\textwidth, height=0.25\textheight, valign=center]
        \Large \textbf{#1}
    \end{tcolorbox}
    \vspace{0.5cm}
}

\definecolor{codegreen}{rgb}{0,0.6,0}
\definecolor{codegray}{rgb}{0.5,0.5,0.5}
\definecolor{codepurple}{rgb}{0.58,0,0.82}
\definecolor{backcolour}{rgb}{0.95,0.95,0.92}

\lstdefinestyle{mystyle}{
    backgroundcolor=\color{backcolour},   
    commentstyle=\color{codegreen},
    keywordstyle=\color{magenta},
    numberstyle=\tiny\color{codegray},
    stringstyle=\color{codepurple},
    basicstyle=\ttfamily\footnotesize,
    breakatwhitespace=false,         
    breaklines=true,                 
    captionpos=b,                    
    keepspaces=true,                 
    numbers=left,                    
    numbersep=5pt,                  
    showspaces=false,                
    showstringspaces=false,
    showtabs=false,                  
    tabsize=2
}
\lstset{style=mystyle}

\newenvironment{lpu}{
  \sffamily
  \fontsize{14pt}{16pt}\selectfont
  \lstset{basicstyle=\fontsize{13pt}{15pt}\ttfamily}
}{
  \normalfont
  \lstset{basicstyle=\normalsize\ttfamily}
}


\usepackage{tikz}
\usetikzlibrary{positioning}

\usepackage{tabularx}
\usepackage{colortbl}

\usepackage{booktabs}
\usepackage{longtable}

\usepackage{xcolor}
\usepackage[framemethod=TikZ]{mdframed}

% Stil für Aufgabennummer
\newtcbox{\aufgabenummerbox}[1][]{
  on line,
  colback=gray!20,
  colframe=black,
  boxrule=0.8pt,
  arc=2pt,
  outer arc=2pt,
  boxsep=1pt,
  left=4pt,
  right=4pt,
  top=2pt,
  bottom=2pt,
  fontupper=\bfseries,
  #1
}

% Umgebung für Aufgaben
\newenvironment{aufgabe}[1]
{
\begin{mdframed}[
  backgroundcolor=blue!5,
  linecolor=blue!40!black,
  linewidth=1pt,
  roundcorner=4pt,
  innertopmargin=6pt,
  innerbottommargin=6pt,
  innerleftmargin=10pt,
  innerrightmargin=10pt,
  skipabove=12pt,
  skipbelow=12pt
]
\aufgabenummerbox{Aufgabe #1}\vspace{6pt}
}
{
\end{mdframed}
}

% Umgebung für Aufgaben
\newenvironment{theorie}
{
\begin{mdframed}[
  backgroundcolor=orange!5,
  linecolor=blue!40!black,
  linewidth=1pt,
  roundcorner=4pt,
  innertopmargin=6pt,
  innerbottommargin=6pt,
  innerleftmargin=10pt,
  innerrightmargin=10pt,
  skipabove=12pt,
  skipbelow=12pt
]
\aufgabenummerbox{Theorie}\vspace{6pt}
}
{
\end{mdframed}
}


\usepackage{svg}


\usepackage{graphicx}
\usepackage{subcaption}
\usepackage[export]{adjustbox}  % für zentrierte Bilder mit valign



\definecolor{featuregreen}{RGB}{220, 255, 220}
\definecolor{targetred}{RGB}{255, 230, 230}

\usepackage{tcolorbox}
\usepackage{booktabs}

\newenvironment{hinweis}{
  \begin{mdframed}[backgroundcolor=yellow!10, linecolor=yellow!50!black, linewidth=1pt, roundcorner=5pt]
  \textbf{Hinweis:}
}{
  \end{mdframed}
}


\usepackage{mdframed}
\definecolor{hellgrau}{gray}{0.95}

\newmdenv[
  backgroundcolor=hellgrau,
  linecolor=black,
  linewidth=0.8pt,
  roundcorner=4pt,
  innertopmargin=1em,
  innerbottommargin=1em,
  innerleftmargin=1em,
  innerrightmargin=1em,
  font=\rmfamily
]{artikelbox}

\newcommand{\artikelheader}[1]{\textbf{\Large #1}\par\vspace{0.5em}}
\newcommand{\artikelsubheader}[1]{\vspace{0.5em}\textbf{\normalsize #1}\par\vspace{0.3em}}